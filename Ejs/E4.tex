\item Hallar un hiperespacio de $\R^2$ y uno de $\R^3$. Interpretar gráficamente.
    \begin{mdframed}[style=s]
        De la \textbf{Definición 6.6} sabemos que para que un subespacio sea un hiperespacio es necesario que $dimW=n-1$
        \begin{itemize}
            \item En $\R^2$, cualquier recta que pase por el origen me serviría. En particular el eje de ordenadas cumple con los requisitos. Además, se puede pensar que $W$ es el núcleo de un funcional lineal $f:\R^2\to\R$, osea que 
            \begin{tightcenter}
                $f(x,y)=0\quad\forall (x,y):x=0$
            \end{tightcenter}
            Como un vector de $\R^2$ se escribe $(x,y)=x(1,0)+y(0,1)\to f(x,y)=x f(1,0)+y f(0,1)\to f(x,y)=x f(1,0)$, siendo $f(1,0)=\alpha\in\R\to f(x,y)=\alpha x$ es un funcional lineal que cumple las características mencionadas. 
            \item En $\R^3$, cualquier plano que contenga al origen cumple la función.
        \end{itemize}
    \end{mdframed}