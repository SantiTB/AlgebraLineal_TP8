\item Consideremos los siguientes vectores de $\R^3$:
    \begin{center}
        $v_1=(-1,-1,0); \quad v_2=(0,1,-2);\quad v_3=(1,0,1)$
    \end{center}
    \begin{enumerate}
        \item Probar que forman una base de $\R^3$ y hallar su base dual.
            \begin{mdframed}[style=s]
                $\vec{0}=x_1(-1,-1,0)+x_2(0,1,-2)+x_3(1,0,1)\to \begin{cases}
                    -x_1+x_3=0\\
                    -x_1+x_2=0\\
                    -2x_2+x_3=0
                \end{cases}\to\begin{cases}
                    x_1=x_3\\
                    x_1=x_2\\
                    2x_2=x_3
                \end{cases}\to x_1,x_2,x_3=0\to$ son li.\\
                Como $v_1,v_2,v_3$ son vectores li de $\R^3$, estos forman una base de $\R^3$\\
                Sea $v=(x,y,z)\in\R^3,(x,y,z)=\alpha(-1,-1,0)+\beta(0,1,-2)+\gamma(1,0,1)\to\begin{cases}
                    \alpha=x-2y-z\\
                    \beta=x-y-z\\
                    \gamma=2x-2y-z
                \end{cases}$\\
                $\begin{cases}
                    f_1(x,y,z)=\alpha f_1(-1,-1,0)+\beta f_1(0,1,-2)+\gamma f_1(1,0,1)=x-2y-z\\
                    f_2(x,y,z)=\alpha f_2(-1,-1,0)+\beta f_2(0,1,-2)+\gamma f_2(1,0,1)=x-y-z\\
                    f_3(x,y,z)=\alpha f_3(-1,-1,0)+\beta f_3(0,1,-2)+\gamma f_3(1,0,1)=2x-2y-z
                \end{cases}\to B^*=\{f_1,f_2,f_3\}$
            \end{mdframed}
        \item Supongamos que $f:\R^3\to\R$ es un funcional lineal tal que
            \begin{center}
                $f(v_1)=1;\quad f(v_2)=-f(v_1);\quad f(v_3)=3$
            \end{center}
            Hallar explícitamente a $f(x,y,z)$ para todo $(x,y,z)\in\R^3$.¿Cuánto vale $f(3,4,0)$?
            \begin{mdframed}[style=s]
                La \textbf{Ecuación 6.3} nos dice que 
                \begin{center}
                    $f=\sum_{i=1}^n f(b_i)f_i$\\
                    $\to f(x,y,z)=x-2y-z+(-1)(x-y-z)+3(2x-2y-z)$\\
                    $\to f(x,y,z)=6x-7y-3z$\\
                    $\to f(3,4,0)=-10$
                \end{center}
            \end{mdframed}
        \item Hallar un funcional lineal $g:\R^3\to\R$ tal que
            \begin{center}
                $g(v_1)=g(v_2)=0;\quad g(v_3)\neq 0$
            \end{center}
            \begin{mdframed}[style=s]
                Propongo $g(v_3)=1\to g(x,y,z)=0(x-2y-z)+0(x-y-z)+2x-2y-z\to g(x,y,z)=2x-2y-z$
            \end{mdframed}
    \end{enumerate}